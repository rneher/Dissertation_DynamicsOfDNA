\chapter*{Danksagung}
\addcontentsline{toc}{chapter}{\protect Danksagung}

Zu vorderst geb\"uhrt mein Dank Prof.~Ulrich Gerland f\"ur die vorbildliche Betreuung und die interessante
Thematik der Arbeit. Insbesondere m\"ochte ich mich f\"ur seine unerm\"udlichen Bem\"uhungen,
mich f\"ur die faszinierende und verwirrende Welt der Biologie zu begeistern, bedanken. Es hat am Ende 
doch noch funktioniert. Ich hatte w\"ahrend meiner Doktorarbeit alle Freiheit die ich mir w\"unschen konnte
und gleichzeitig immer die M\"oglichkeit Probleme zu er\"ortern oder um Rat zu fragen. 

Desweiteren m\"ochte ich mich bei Wolfram M\"obius f\"ur die fruchtbare Zusammenarbeit zur Dynamik 
von Nukleosomen bedanken. Dynamik von Nukleosomen war das Thema Wolframs Diplomarbeit, die ich anfangs
als Diskussionspartner begleiten durfte und die sich mehr und mehr in ein gemeinsames Projekt entwickelte. 

Julia Morfill und Ferdinand K\"uhner haben sich getraut w\"ahrend der Experimente einen 
Theoretiker ins Labor zu lassen und geduldig meinen bisweilen abwegigen Ideen zugeh\"ort. 
Daf\"ur m\"ochte ich mich 
herzlich bedanken, denn mir hat es viel Spa\ss {} gemacht. In diesem Zusammenhang geb\"uhrt auch Prof.~Hermann Gaub mein Dank, an dessen Lehrstuhl diese Experimente durchgef\"uhrt wurden.

Diese Doktorarbeit wurde in erster Linie am Lehrstuhl von Prof.~Erwin Frey erstellt, dem ich an dieser
Stelle einerseits f\"ur die vorhandene Infrastruktur, vor allem aber auch f\"ur fortw\"ahrende Unterst\"utzung
und hilfreiche Diskussionen danken m\"ochte. Auch bei allen \"ubrigen Mitglieder des Lehrstuhls, insbesondere
bei Georg Fritz, mit dem ich \"uber Jahre ein B\"uro teilte, m\"ochte ich mich herzlich f\"ur die angenehme 
und anregende Arbeitsatmosph\"are bedanken. Prof.~Herbert Wagner bin ich f\"ur unz\"ahlige gute Ratschl\"age
und interessante Diskussionen zu Dank verpflichtet. F\"ur das Korrekturlesen der Arbeit danke ich meinem Mitbewohner
Tim Liedl und Wolfram M\"obius. 

Finanzielle Unterst\"utzung erhielt ich \"uber das Emmy-Noether-Stipendium von Ulrich Gerland von der DFG
sowie vom Internationalen Doktoranden Kolleg Nanobiotechnologie (IDK-NBT). Das IDK-NBT hat unter anderem
in gro\ss\"ugiger Weise viele meiner Forschungs- und Fortbildungsreisen bezahlt. W\"ahrend des 
vergangen Jahres habe ich mehrere Wochen am Center for Theoretical Biological Physics 
an der UCSD und am Kavli Institute for Theoretical Physics in Santa Barbara, Kalifornien verbracht. 
Ich hab w\"ahrend beider Aufenthalte viel gelernt und m\"ochte beiden Zentren f\"ur die Gastfreundschaft
danken. Ohne die finanzielle Unterst\"utzung und die anregenden Reisen
w\"are meine Doktorarbeit sicherlich nicht so fruchtbar gewesen. 