

\chapter*{Abstract}
\addcontentsline{toc}{chapter}{\protect Abstract}
DNA is not a rigid entity, but a highly dynamic molecule. The dynamics of DNA on different
length scales is the objective of this thesis. The first part of this thesis
addresses the dynamics of the base pairing patterns of DNA, the sequence of which is a multi-fold
repetition of a short motif. In the second part of this thesis, we discuss the dynamics of chromatin.

Repetitions of short motifs of one to six bases are very common in eukaryotic genomes and the 
number of repeated units changes extraordinarily fast
from generation to generation. The rate of such contractions or deletions is orders of magnitudes
larger than the rate of ordinary point mutations. This hyper-variability of repetitive DNA has a number 
of implications in biology and is the cause of certain human hereditary diseases. 
The reason why repetitive DNA mutates so rapidly is related to the fact, that two complementary
strands with repetitive sequence can bind to each other, even when shifted relative to each other.
Locally shifted binding is called \emph{DNA-slippage} and leads to errors during DNA replication. 
We studied the dynamics of DNA-slippage theoretically and suggest experiments that probe
DNA-slippage in single DNA molecules. The propagation of small bulge loops in the double
helix of repetitive DNA allows the two strands to move relative to each other. Application of 
a shear force to repetitive DNA should therefore induce a strand motion. The 
bulge loops are produced by DNA slippage at the ends of the double strand. We show, that the
bulge loop production rate and hence the relative velocity of the two strands depends sensitively
on the applied shear force. We uncover four dynamical regimes, where the rupture times scale
differently with the system size. Ferdinand K\"uhner and Julia Morfill from the lab of Prof.~H.E.~Gaub
succeeded in measuring force induced DNA-slippage in single molecules using an atomic force
microscope \cite{Kuehner_BiophysJ_07}. In addition to its biological relevance, repetitive DNA 
has intriguing mechanical properties that might find applications in nanotechnology. Repetitive DNA
acts as a contractile visco-elastic element, the characteristics of which can be programmed by
its length and sequence composition. Rare point mutations that interrupt the repetitive sequence allow
to delay the response in a controlled manner.

The second part of this thesis addresses the dynamics of nucleosomes, which are the elementary 
packing units of chromatin. Eukaryotic cells compactify their genome to make it fit into the cell's nucleus.
Nevertheless, the cell has to access the information in the DNA. Since most proteins cannot bind 
to DNA buried in nucleosomes, the question how often and how rapidly a particular 
stretch of DNA detaches from the nuclesome is of great biological relevance. 
A nucleosome consists of a cylindrical protein core with 6~nm in diameter. The DNA is wrapped
around this protein cylinder approximately 1.7 times. Recent experiments measured the rates,
at which the DNA detaches and attaches partly from the protein core \cite{Li_NatureStructMolBio_05, Tomschik_PNAS_05}.
We studied the dynamics of DNA wrapping and unwrapping in single nucleosomes theoretically.
%The wrapping and unwrapping is a thermally activated barrier crossing process. 
We show, that the small but finite flexibility of the DNA drastically enhances the rates of the 
wrapping and unwrapping kinetics. The rich phenomenology and the ubiquity of similar
processes in biology motivated us to study transition that involve the rotation of flexible
lever-like object in more detail. The transition rate displays an optimum at an intermediate 
stiffness. The optimal stiffness parameters could be realized by evolution in biological macromolecules
such as molecular motors.

In the first chapter of this thesis, I present general features of DNA such as its chemical 
composition, its structure, its thermodynamics and its mechanical properties. The second
chapter is on the dynamics of repetitive DNA. First, we discuss the biological significance of repetitive DNA
and existing experimental evidence for DNA-slippage. Then we present our theoretical analysis
and our publications on repetitive DNA. The third chapter is on chromatin dynamics. To begin with,
chromatin structure and its implication for gene regulation in eukaryotes are discussed. This is 
followed by a discussion of recent experiments on single nuclesome dynamics, our theoretical
work, and our publications.
