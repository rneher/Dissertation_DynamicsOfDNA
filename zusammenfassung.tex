\chapter*{Zusammenfassung}
\addcontentsline{toc}{chapter}{\protect Zusammenfassung}

DNA ist keine steife und festgef\"ugte Einheit, sondern \"andert fortlaufend ihre Konformation. Die Dynamik
von DNA auf unterschiedlichen L\"angenskalen ist Thema dieser Dissertation. 
Der erste Teil der Arbeit befasst sich mit der Dynamik der Basenpaarung zwischen 
zwei DNA Str\"angen, deren Sequenz eine mehrfache Wiederholung eines kurzen Motivs ist. 
Im zweiten Teil der Arbeit wird die Dynamik von Chromatin, d.h.~DNA, die mit Hilfe von Proteinen
in Chromosome gepackt ist, diskutiert. 

Vielfache Wiederholungen eines kurzen Motivs von ein bis sechs Basen sind sehr
h\"aufig in eukaryotischen Genomen und haben die erstaunliche Eigenschaft, dass sich die Zahl der wiederholten Einheiten
ausserordentlich schnell von Generation zu Generation \"andert. Die Rate solcher Mutationen 
\"ubersteigt die von Punktmutationen um mehrere Gr\"o\ss{}enordnun\-gen. 
Diese Hypervariabilit\"at repetitiver Sequenzen hat eine Reihe 
von biologischen Konsequenzen und ist  unter anderem f\"ur einige menschliche Erbkrankheiten verantwortlich. 
Repetitive DNA mutiert um so vieles schneller als gew\"ohnliche DNA, da die beiden Str\"ange
gegeneinander versetzt binden k\"onnen und dadurch Fehler bei der DNA Replikation auf\-treten.   
Dieses versetzte Binden heisst \emph{DNA-slippage}. Wir haben die
Dynamik von DNA-slippage theoretisch untersucht und Experimente vorgeschlagen, mit denen  
DNA-slippage in einzelnen Molek\"ulen detektiert werden kann. 
Zwei kurze repetitive DNA Str\"ange k\"onnen sich durch Propagation von Defekten
gegeneinander bewegen und daher durch eine Scher\-kraft aneinander entlang gezogen werden. 
Die Defekte werden durch DNA-slippage an den Enden des Doppelstrangs erzeugt. 
Die Rate, mit der Defekte produziert werden und damit die Geschwindigkeit, mit der die Str\"ange
sich gegeneinander bewegen, h\"angt sehr empfindlich von der angelegten Kraft ab. 
Unsere theoretische Analyse hat gezeigt, dass
es vier dynamische Regime gibt, in denen die typischen Abrisszeiten unterschiedlich mit 
L\"ange des Molek\"uls anwachsen. Ferdinand K\"uhner und Julia Morfill aus dem Labor von 
Prof.~H.E.~Gaub haben k\"urzlich mit Hilfe eines Kraftmikroskops (AFM) experimentell gezeigt, 
dass DNA-slippage tats\"achlich 
durch Scherkr\"afte ausgel\"ost werden kann \cite{Kuehner_BiophysJ_07}. 
\"Uber die biologische Relevanz hinaus k\"onnte repetitive DNA auch Anwendungen in der Nanotechnologie 
finden, denn sie verh\"alt sich wie ein kontraktiles visko-elastisches Element. Die Kenngr\"o\ss{}en
eines solchen Elements k\"onnen durch Wahl der Sequenz und der L\"ange der Str\"ange 
programmiert werden. Durch einzelne Punktmutationen, die die Periodizit\"at der Sequenz unterbrechen, 
kann die mechanische Antwort des Systems gezielt verz\"ogert werden.

Der zweite Teil der Dissertation behandelt die Dynamik des elementaren Baustein von Chromatin, dem Nukleosom.
Damit das Genom eukaryotischer Zellen in den Zellkern passt, ist die DNA dicht gepackt. 
Trotzdem muss die in der DNA gespeicherte Information f\"ur die Zelle zug\"anglich sein. 
Daher ist die Frage, wie oft und wie lang ein bestimmer Teil der DNA sich von dem Nukleosom l\"ost
 von gro\ss{}er biologischer Relevanz.
Ein Nukleosom besteht aus einem zylindrischen Proteinkomplex
mit einen Durchmesser von ca.~6~nm, um den die DNA in etwa 1.7 mal gewickelt ist. 
Es wurde k\"urzlich experimentell gezeigt \cite{Li_NatureStructMolBio_05,Tomschik_PNAS_05}, 
dass sich Teile der DNA eines Nukleosoms auf eine Skala von Millisekunden bis Sekunden
vom Proteinzylinder l\"osen. Diese Dynamik k\"onnte Teil des Mechanismus sein, mit Hilfe dessen
die Zelle Zugang zu kompaktifizierter DNA erlangt.  Komplement\"ar zu diesen Experimenten haben
wir Nukleosom-Dynamik theoretisch untersucht. Unsere Studien haben gezeigt, dass
die wenn auch kleine Flexibilit\"at der DNA einen au\ss{}ordentlich gro\ss{}en Einfluss auf die Dynamik 
solcher DNA-Protein Komplexe hat. Der wesentliche Prozess des Auf- und Abwickelns ist  thermisch
aktiviertes \"Uberqueren einer Potentialbarriere, in dessen Verlauf sich die DNA reorientiert. 
Die reichhaltige Ph\"anomenologie und die Allgegenw\"artigkeit solcher Prozesse hat uns 
motiviert thermisch aktiviertes \"Uberqueren einer Potentialbarriere gekoppelt an die 
Rotation eines flexiblen Arms genauer zu studieren. Die Rate f\"ur das \"Uberqueren der Barriere
wird maximal bei einer intermedi\"aren Steifigkeit. 
Solche optimalen Parameter k\"onnten in biologischen Makromolek\"ulen
wie z.B.~molekularen Motoren realisiert sein.

Im ersten Kapitel dieser Arbeit werden die chemische Zusammensetzung von DNA, ihre Struktur, sowie
ihre thermodynamischen und mechanischen Eigenschaften diskutiert. Das zweite Kapitel
befasst sich mit der Dynamik repetitiver DNA Sequenzen. Zu Beginn wird die biologische Rolle
repetitiver DNA und ihre Verbindung zu menschlichen Erbkrankheiten vorgestellt. Dann diskutiere
ich die Grundz\"uge unserer theoretischen Arbeit sowie die erste experimentelle Best\"atigung von 
DNA-slippage, gefolgt von unseren Publikationen zu diesem 
Themenkomplex. Das dritte Kapitel befasst sich mit der Dynamik von Chromatin. Nach der
Struktur von Chromatin werden die Experimente zur Nukleosom-Dynamik diskutiert und im 
Anschluss unsere theoretische Arbeit und unsere Publikationen vorgestellt. 
